% The Introduction should explain the principal motivation for the project. Show how the work fits into the broad 
% area of surrounding Computer Science and give a brief survey of previous related work. It should generally be 
% unnecessary to quote at length from technical papers or textbooks. If a simple bibliographic reference is 
% insufficient, consign any lengthy quotation to an appendix.

% ~500 words

\documentclass[final,dissertation.tex]{subfiles}
\begin{document}

\chapter{Introduction}

Loopix \cite{piotrowska2017loopix} is an anonymous communication system that provides medium-latency, low-bandwidth communication with strong anonymity. Loopix is traffic analysis resistant to both active attackers and global passive adversaries. This is achieved by utilising cover traffic and Poisson mixing which adds brief independent message delays.

I have successfully created a Java implementation of the Loopix client that is compatible with the original Python implementation. I also provide an analysis of my client and Loopix by measuring bandwidth and latency overhead and discussed the viability of Loopix on mobile devices.

\section{Motivation}


Anonymous communication systems such as Tor \cite{dingledine2004tor} allows users to communicate without revealing the identities of users to third parties. These are becoming increasingly important in the current age of pervasive data collection, surveillance and censorship, allowing for privacy and anonymity when parties are trying to collect as much data as possible on individuals, either for commercial exploitation or government surveillance. Such systems enable citizens in oppressive regimes to access censored material. Whistleblowers use anonymous communications to securely communicate with media organisations to protect their identity. Non-governmental organisations such as human rights activists can use such systems to avoid persecution while conducting activities. There are many use cases for anonymous communication systems, and these are just a few of possible uses.

However, the most widely used anonymous communication system, Tor, is vulnerable to attacks such as traffic correlation by a global passive adversary and corrupt nodes performing active attacks to deanonymise users. A global passive adversary is an adversary that is able to observe all traffic traversing all communication channels. This ability allows such an adversary to correlate the sender's and recipient's traffic patterns and link the two together. Government agencies such as the National Security Agency (NSA) and Government Communications Headquarters (GCHQ) have already demonstrated the ability to deanonymise a small fraction of Tor users \cite{torstinks}. Alternative systems that are not vulnerable to a global passive adversary tend to be high-latency, low-bandwidth, which severely limits possible applications due to the anonymity trilemma of choosing two out of strong anonymity, low bandwidth overhead, or low latency \cite{das2017anonymity}.

The medium-latency property Loopix should allow for communication applications such as instant messaging and email which are not very latency sensitive. A useful feature of Loopix is the ability to store messages for offline clients. Clients running mobile devices tend to have sporadic connections, which means such clients are not always connected to the network.

\section{Related Work}

Anonymous communication systems can be broadly categorised into either connection-based onion routing systems or message-based mix networks. The most popular architecture is onion routing as used in Tor, with various systems extending this paradigm. Message-based mix networks such as Mixminion \cite{danezis2003mixminion}, which Loopix is loosely based upon, on the other hand, are currently unfashionable due to the high-latencies involved.

The plethora of anonymous communication system designs shows how difficult it is to design an anonymous communication system with acceptable trade-offs for many different applications. Some require many clients to provide anonymity, which is not ideal as users would only use the system if it provides anonymity. Loopix attempts to cater for medium-to high-latency applications such as instant messaging and email, and not target very low-latency applications such as web browsing. Loopix's parameters can also be adjusted according to the number of participating users to maintain anonymity even when there are very few users.

The Loopix authors have already completed a rigorous analysis of Loopix and have a working implementation in Python. This meant I was able to use similar evaluation methodologies to the original paper \cite{piotrowska2017loopix}, and their open source Python implementation\footnote{\url{https://github.com/UCL-InfoSec/loopix}} was crucial in developing my own implementation.

Loopix has since been worked on and improved by the PANORAMIX project, and the end result is a more robust mix network Katzenpost\footnote{https://katzenpost.mixnetworks.org/}. Katzenpost is based on Loopix, and adds a consensus-based public key infrastructure, reliable message delivery, and single-use replies.

\end{document}