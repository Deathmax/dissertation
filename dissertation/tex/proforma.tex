\documentclass[final,dissertation.tex]{subfiles}
\begin{document}

\pagestyle{plain}

\chapter*{Proforma}

{\large
	\begin{tabular}{ll}
		Name:               & \bf Jun Siang Cheah                               \\
		College:            & \bf Christ's College                              \\
		Project Title:      & \bf An implementation and evaluation of Loopix,   \\ 
		& \bf an anonymous communication system                                 \\
		Examination:        & \bf Computer Science Tripos -- Part II, July 2018 \\
		Word Count:         & \bf 10203\footnotemark                              \\
		Project Originator: & Dr Alastair Beresford                             \\
		Supervisor:         & Dr Alastair Beresford                             \\ 
	\end{tabular}
}

\footnotetext{Computed using \verb|perl texcount.pl -total chapter_{1,2,3,4,5}.tex| from \href{http://app.uio.no/ifi/texcount/}{http://app.uio.no/ifi/texcount/}}

\section*{Original Aims of the Project}

% What is Loopix, background

Loopix is a communication system that allows users to anonymously communicate, without revealing that two users are communicating. This project aimed to implement a Loopix client library in Java, and a demonstration application running on top of the library. The implementation was to be evaluated in terms of various performance metrics such as latency and bandwidth overheads.

\section*{Work Completed}

% Sell it better, more numbers

I have created a working, binary compatible Loopix client in Java. A latency overhead due to processing of about \SI{20}{\milli\second} was measured. This is faster than the Python implementation by \SI{2}{\milli\second}. I have also analysed the viability of running a Loopix client on mobile devices, with results showing the trade-off between delays necessary for maintaining anonymity and resource usage such as bandwidth and battery usage may be too large to reconcile for many applications.

\section*{Special Difficulties}

None

\end{document}